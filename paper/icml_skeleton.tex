% ==================================================================
% ICML 2025 Manuscript Skeleton (submission mode, anonymized)
% Followed closely to the official template and guidelines.
% Switch to camera-ready by using: \usepackage[accepted]{icml2025}
% ==================================================================
\documentclass{article}

% --- ICML style ---
% NOTE: Do NOT add the [accepted] option for initial (anonymous) submission.
\usepackage{icml2025}

% --- Recommended but safe packages (do not change fonts/margins) ---
\usepackage{microtype}
\usepackage{graphicx}
\usepackage{booktabs}
\usepackage{amsmath,amssymb,amsthm}
\usepackage{mathtools}

% --- Theorem-like environments (optional; remove if unused) ---
\theoremstyle{plain}
\newtheorem{theorem}{Theorem}[section]
\newtheorem{lemma}[theorem]{Lemma}
\newtheorem{proposition}[theorem]{Proposition}
\newtheorem{corollary}[theorem]{Corollary}
\theoremstyle{definition}
\newtheorem{definition}[theorem]{Definition}
\theoremstyle{remark}
\newtheorem{remark}[theorem]{Remark}
\newtheorem{assumption}[theorem]{Assumption}

% --- Running title for camera-ready (ignored in submission PDF header) ---
\icmltitlerunning{Short Title}

\begin{document}

% ================================================================
% Title & Author Block (authors will be hidden in submission mode)
% ================================================================
\twocolumn[
\icmltitle{<Full Paper Title>}

% It is OK to specify authors/affiliations here;
% they are NOT printed unless the style option [accepted] is used.
\begin{icmlauthorlist}
\icmlauthor{First A. Author}{xxx}
\icmlauthor{Second B. Author}{yyy}
\icmlauthor{Third C. Author}{zzz}
\end{icmlauthorlist}

\icmlaffiliation{xxx}{Affiliation One, City, Country}
\icmlaffiliation{yyy}{Affiliation Two, City, Country}
\icmlaffiliation{zzz}{Affiliation Three, City, Country}

\icmlcorrespondingauthor{First A. Author}{first.last@institution.edu}

% Keywords appear in PDF metadata
\icmlkeywords{Machine Learning, ICML}

\vskip 0.3in
]

% Use \icmlEqualContribution if needed; otherwise leave the braces empty.
\printAffiliationsAndNotice{} % \printAffiliationsAndNotice{\icmlEqualContribution}

% ==================
% Abstract (required)
% ==================
\begin{abstract}
% Single paragraph; ideally 4--6 sentences; anonymized.
\end{abstract}

% =============
% Main Sections
% =============
\section{Introduction}
% Motivation, summary of contributions, roadmap.
% Avoid self-identifying statements during double-blind review.

\section{Related Work}
% Position your contributions relative to prior work.

\section{Preliminaries and Problem Setup}
% Formal definitions, notation, and task/setting description.

\section{Method}
% Core approach; algorithms; complexity; implementation notes if needed.

\section{Theory (Optional)}
% State theorems, lemmas, and proofs if applicable.

\section{Experiments}
% Leave placeholders for datasets, baselines, metrics, and protocols.
% Figures/tables should be legible and follow ICML placement rules.

\section{Limitations}
% Briefly discuss known limitations or failure modes.

\section{Conclusion}
% Recap contributions and takeaways.

% ========================
% Impact Statement (must be BEFORE References; required)
% ========================
\section*{Impact Statement}
% Concisely discuss potential broader impacts (ethical/societal), positive and negative.

% ========================
% Acknowledgements (camera-ready only; DO NOT include in submission)
% ========================
% \section*{Acknowledgements}
% We thank ... (funding, helpful discussions, etc.).

% ========================
% References (APA style via natbib + icml2025.bst)
% ========================
\bibliographystyle{icml2025}
\bibliography{references}

% ========================
% Appendices (after References)
% ========================
\appendix

\section{Additional Experimental Details}
% Datasets, hyperparameters, training schedules, compute budget.

\section{Proofs}
% Technical lemmas and theorem proofs.

\section{Extra Results}
% Ablations, qualitative examples, extended tables/figures.

\end{document}
